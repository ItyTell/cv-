\documentclass[10pt, letterpaper]{article}


% Packages:
\usepackage[
    ignoreheadfoot, % set margins without considering header and footer
    top=2 cm, % seperation between body and page edge from the top
    bottom=2 cm, % seperation between body and page edge from the bottom
    left=2 cm, % seperation between body and page edge from the left
    right=2 cm, % seperation between body and page edge from the right
    footskip=1.0 cm, % seperation between body and footer
    % showframe % for debugging 
]{geometry} % for adjusting page geometry
\usepackage{titlesec} % for customizing section titles
\usepackage{tabularx} % for making tables with fixed width columns
\usepackage{array} % tabularx requires this
\usepackage[dvipsnames]{xcolor} % for coloring text
\definecolor{primaryColor}{RGB}{0, 0, 0} % define primary color
\usepackage{enumitem} % for customizing lists
\usepackage{fontawesome5} % for using icons
\usepackage{amsmath} % for math
\usepackage[
    pdftitle={John Doe's CV},
    pdfauthor={John Doe},
    pdfcreator={LaTeX with RenderCV},
    colorlinks=true,
    urlcolor=primaryColor
]{hyperref} % for links, metadata and bookmarks
\usepackage[pscoord]{eso-pic} % for floating text on the page
\usepackage{calc} % for calculating lengths
\usepackage{bookmark} % for bookmarks
\usepackage{lastpage} % for getting the total number of pages
\usepackage{changepage} % for one column entries (adjustwidth environment)
\usepackage{paracol} % for two and three column entries
\usepackage{ifthen} % for conditional statements
\usepackage{needspace} % for avoiding page brake right after the section title
\usepackage{iftex} % check if engine is pdflatex, xetex or luatex
% Ensure that generate pdf is machine readable/ATS parsable:
\ifPDFTeX
    \input{glyphtounicode}
    \pdfgentounicode=1
    \usepackage[T1]{fontenc}
    \usepackage[utf8]{inputenc}
    \usepackage{lmodern}
\fi

\usepackage{charter}
\usepackage[english,ukrainian]{babel}
\usepackage{tempora}
\usepackage{hyperref}
% Settings:
\raggedright
\AtBeginEnvironment{adjustwidth}{\partopsep0pt} % remove space before adjustwidth environment
\pagestyle{empty} % no header or footer
\setcounter{secnumdepth}{0} % no section numbering
\setlength{\parindent}{0pt} % no indentation
\setlength{\topskip}{0pt} % no top skip
\setlength{\columnsep}{0.15cm} % set column seperation
\pagenumbering{gobble} % no page numbering

\titleformat{\section}{\needspace{4\baselineskip}\bfseries\large}{}{0pt}{}[\vspace{1pt}\titlerule]

\titlespacing{\section}{
    % left space:
    -1pt
}{
    % top space:
    0.3 cm
}{
    % bottom space:
    0.2 cm
} % section title spacing

\renewcommand\labelitemi{$\vcenter{\hbox{\small$\bullet$}}$} % custom bullet points
\newenvironment{highlights}{
    \begin{itemize}[
        topsep=0.10 cm,
        parsep=0.10 cm,
        partopsep=0pt,
        itemsep=0pt,
        leftmargin=0 cm + 10pt
    ]
}{
    \end{itemize}
} % new environment for highlights


\newenvironment{highlightsforbulletentries}{
    \begin{itemize}[
        topsep=0.10 cm,
        parsep=0.10 cm,
        partopsep=0pt,
        itemsep=0pt,
        leftmargin=10pt
    ]
}{
    \end{itemize}
} % new environment for highlights for bullet entries

\newenvironment{onecolentry}{
    \begin{adjustwidth}{
        0 cm + 0.00001 cm
    }{
        0 cm + 0.00001 cm
    }
}{
    \end{adjustwidth}
} % new environment for one column entries

\newenvironment{twocolentry}[2][]{
    \onecolentry
    \def\secondColumn{#2}
    \setcolumnwidth{\fill, 4.5 cm}
    \begin{paracol}{2}
}{
    \switchcolumn \raggedleft \secondColumn
    \end{paracol}
    \endonecolentry
} % new environment for two column entries

\newenvironment{threecolentry}[3][]{
    \onecolentry
    \def\thirdColumn{#3}
    \setcolumnwidth{, \fill, 4.5 cm}
    \begin{paracol}{3}
    {\raggedright #2} \switchcolumn
}{
    \switchcolumn \raggedleft \thirdColumn
    \end{paracol}
    \endonecolentry
} % new environment for three column entries

\newenvironment{header}{
    \setlength{\topsep}{0pt}\par\kern\topsep\centering\linespread{1.5}
}{
    \par\kern\topsep
} % new environment for the header

\newcommand{\placelastupdatedtext}{% \placetextbox{<horizontal pos>}{<vertical pos>}{<stuff>}
  \AddToShipoutPictureFG*{% Add <stuff> to current page foreground
    \put(
        \LenToUnit{\paperwidth-2 cm-0 cm+0.05cm},
        \LenToUnit{\paperheight-1.0 cm}
    ){\vtop{{\null}\makebox[0pt][c]{
        \small\color{gray}\textit{Last updated in July 2024}\hspace{\widthof{Last updated in July 2024}}
    }}}%
  }%
}%

% save the original href command in a new command:
\let\hrefWithoutArrow\href

% new command for external links:

\newcommand{\AND}{\unskip
    \cleaders\copy\ANDbox\hskip\wd\ANDbox
    \ignorespaces
}

\hypersetup{
    colorlinks=true,
    linkcolor=black,
    filecolor=black,      
    urlcolor=black,
    pdftitle={CV-Kolomiiets},
    pdfpagemode=FullScreen,
    }



\begin{document}
    \newsavebox\ANDbox
    \sbox\ANDbox{$|$}

    \begin{header}
        \fontsize{25 pt}{25 pt}\selectfont Микола Коломієць

        \vspace{5 pt}

        \normalsize
        \mbox{Київ, Україна}%
        \kern 5.0 pt%
        \AND%
        \kern 5.0 pt%
        \mbox{\hrefWithoutArrow{mailto:nickkolomietsdooor@gmail.com}{nickkolomietsdooor@gmail.com}}%
        \kern 5.0 pt%
        \AND%
        \kern 5.0 pt%
        \mbox{\hrefWithoutArrow{tel:+380 672 570 278}{+380 672 570 278}}%
        \kern 5.0 pt%
        \AND%
        \kern 5.0 pt%
        \mbox{\hrefWithoutArrow{https://itytell.github.io/site/}{Portfolio site}}%
        \kern 5.0 pt%
        \AND%
        \kern 5.0 pt%
        \mbox{\hrefWithoutArrow{www.linkedin.com/in/mycola-kolomiiets-5350961b0}{linkedin}}%
        \kern 5.0 pt%
        \AND%
        \kern 5.0 pt%
        \mbox{\hrefWithoutArrow{https://github.com/ItyTell}{github}}%
    \end{header}

    \vspace{5 pt - 0.3 cm}

    \section{Освіта}
        
        \begin{twocolentry}{
            Вересень 2020 – Червень 2024
        }
            \textbf{Київський нацьональний університет ім. Тараса Шевченка}, факультет комп'ютерних наук, спеціальність прикладна математика \end{twocolentry}

        \vspace{0.10 cm}
        \begin{onecolentry}
            \begin{highlights}
                \item GPA: 89/100
            \end{highlights}
        \end{onecolentry}



    
    \section{Досвід}

        
        \begin{twocolentry}{
            Серпень – Жовтень 2023
        }
            \textbf{Степендіат програми "IRIS-HEP Fellow 2023"}, Прінстонський університет, CERN \end{twocolentry}

        \vspace{0.10 cm}
        \begin{onecolentry}
            Під час програми співпрацював з спеціалістами у галузях розробки програмного запезпечення та data science над проектом, пов'язаним з фізикою високих часток.
        \end{onecolentry}

    \section{Мова}
        

        \begin{samepage}
            \textbf{Англійська, рівень B2}


            Рівень B2, через невелику кількість помилок на ЗНО і на ЄВІ. Маю досвід роботи з іноземними колегами та викладачами. Більшість контенту що споживав 4 роки було англійською мовою, тому розуміння мови на високому рівні проте не було багато розмовної практики.


            \textbf{Українська мова, рідна}

        \end{samepage}

    \section{Навички}

            
        \begin{onecolentry}
            \begin{highlightsforbulletentries}


            \item Python


            В університеті я пройшов річний курс з Python. 
            Потім використовував його для автоматизації розрахунків з інших предметів і створювати особисті проекти, такі як телеграм-бот для надання певної статистики та просту 2-D гру на pygame. Останніми алгоритмами, які я реалізував у Python, були деякі алгоритми кластеризації: k-середні, нечіткий k-means (варіація попереднього) і DBScan. 
            Також використовував python для дипломної роботи з прогнозування шкідливих викидів у атмосферу. 

            \item Математика
            

            Я завершив навчання за спеціальністю «прикладна математика». 
            Я вивчав різні розділи математики, включаючи статистику, теорію ймовірності, чисельні методи та математичний аналіз. 


            \item Глибоке навчання 
            

            Я пройшов курс на Coursera «Спеціалізація глибокого навчання» від Deep Learning.AI і отримав сертифікат у 2023 році. 
            Тоді я погрався з набутими знаннями в Jupyter Notebook, створивши та навчання декількох нейронних мереж. 
            Застосовував нейронні мережі у дипломній роботі та в особистих проектах, також працював з нейронними мережами на степендіальній програмі.

            \item Git, Docker, Linux
            

            Використовую git для більшості своїх проектів та маю досвіт роботи з git та dicker після степендіальної програми. 
            Раніше мав ком'ютер на Linux, зараз використовую Linux переважно через WSL для застосування графічного процесора у tensorflow (основний комп'ютер на Windows).
            
            
            \item C++
            

            Я пройшов річний курс C++ в університеті. 
            Курс був більше про об'єктно-орієнтовану парадигму, але всі приклади та завдання були на С++. 


            \item SQL
            

            В університеті я пройшов піврічний курс SQL. 
            Під час курсу я навчився писати запити різної складності, в тому числі дуже складноі. 
            Курс базувався на Microsoft Access, хоча я пізніше застосував отримані знання, під час роботи над телеграм-ботом на Python, для якого я спочатку хотів використати SQLite, але зупинився на MySQL.
            

            \end{highlightsforbulletentries}
        \end{onecolentry}

\end{document}