\documentclass[10pt, letterpaper]{article}


% Packages:
\input{packages}
% Settings:
\input{settings}

\begin{document}
    \newsavebox\ANDbox
    \sbox\ANDbox{$|$}

    \begin{header}
        \fontsize{25 pt}{25 pt}\selectfont Микола Коломієць

        \vspace{5 pt}

        \normalsize
        \mbox{Київ, Україна}%
        \kern 5.0 pt%
        \AND%
        \kern 5.0 pt%
        \mbox{\hrefWithoutArrow{mailto:nickkolomietsdooor@gmail.com}{nickkolomietsdooor@gmail.com}}%
        \kern 5.0 pt%
        \AND%
        \kern 5.0 pt%
        \mbox{\hrefWithoutArrow{tel:+380 672 570 278}{+380 672 570 278}}%
        \kern 5.0 pt%
        \AND%
        \kern 5.0 pt%
        \mbox{\hrefWithoutArrow{https://itytell.github.io/site/}{Portfolio site}}%
        \kern 5.0 pt%
        \AND%
        \kern 5.0 pt%
        \mbox{\hrefWithoutArrow{www.linkedin.com/in/mycola-kolomiiets-5350961b0}{linkedin}}%
        \kern 5.0 pt%
        \AND%
        \kern 5.0 pt%
        \mbox{\hrefWithoutArrow{https://github.com/ItyTell}{github}}%
    \end{header}

    \vspace{5 pt - 0.3 cm}

    \section{Освіта}
        
        \begin{twocolentry}{
            Вересень 2020 – Червень 2024
        }
            \textbf{Київський нацьональний університет ім. Тараса Шевченка}, факультет комп'ютерних наук, спеціальність прикладна математика \end{twocolentry}

        \vspace{0.10 cm}
        \begin{onecolentry}
            \begin{highlights}
                \item GPA: 89/100
            \end{highlights}
        \end{onecolentry}



    
    \section{Досвід}

        
        \begin{twocolentry}{
            Серпень – Жовтень 2023
        }
            \textbf{Степендіат програми "IRIS-HEP Fellow 2023"}, Прінстонський університет, CERN \end{twocolentry}

        \vspace{0.10 cm}
        \begin{onecolentry}
            Під час програми співпрацював з спеціалістами у галузях розробки програмного запезпечення та data science над проектом, пов'язаним з фізикою високих часток.
        \end{onecolentry}

    \section{Мова}
        

        \begin{samepage}
            \textbf{Англійська, рівень B2}


            Рівень B2, через відсутність помилок на ЗНО і на ЄВІ. Маю досвід роботи з іноземними колегами та викладачами. Більшість контенту що споживав 4 роки було англійською мовою, тому розуміння мови на високому рівні проте не було багато розмовної практики.


            \textbf{Українська мова, рідна}

        \end{samepage}

    \section{Навички}

            
        \begin{onecolentry}
            \begin{highlightsforbulletentries}


            \item Each section title is arbitrary, and each section contains a list of entries.

            \item There are 7 unique entry types: \textit{BulletEntry}, \textit{TextEntry}, \textit{EducationEntry}, \textit{ExperienceEntry}, \textit{NormalEntry}, \textit{PublicationEntry}, and \textit{OneLineEntry}.

            \item Select a section title, pick an entry type, and start writing your section!

            \item \href{https://docs.rendercv.com/user_guide/}{Here}, you can find a comprehensive user guide for RenderCV.


            \end{highlightsforbulletentries}
        \end{onecolentry}

    \section{Projects}



        
        \begin{twocolentry}{
            \href{https://github.com/sinaatalay/rendercv}{github.com/name/repo}
        }
            \textbf{Multi-User Drawing Tool}\end{twocolentry}

        \vspace{0.10 cm}
        \begin{onecolentry}
            \begin{highlights}
                \item Developed an electronic classroom where multiple users can view and simultaneously draw on a "chalkboard" with each person's edits synchronized
                \item Tools Used: C++, MFC
            \end{highlights}
        \end{onecolentry}


        \vspace{0.2 cm}

        \begin{twocolentry}{
            \href{https://github.com/sinaatalay/rendercv}{github.com/name/repo}
        }
            \textbf{Synchronized Calendar}\end{twocolentry}

        \vspace{0.10 cm}
        \begin{onecolentry}
            \begin{highlights}
                \item Developed a desktop calendar with globally shared and synchronized calendars, allowing users to schedule meetings with other users
                \item Tools Used: C\#, .NET, SQL, XML
            \end{highlights}
        \end{onecolentry}


        \vspace{0.2 cm}

        \begin{twocolentry}{
            2002
        }
            \textbf{Operating System}\end{twocolentry}

        \vspace{0.10 cm}
        \begin{onecolentry}
            \begin{highlights}
                \item Developed a UNIX-style OS with a scheduler, file system, text editor, and calculator
                \item Tools Used: C
            \end{highlights}
        \end{onecolentry}



    
    \section{Additional Experience And Awards}



        
        \begin{onecolentry}
            \textbf{Instructor (2003-2005):} Taught 2 full-credit computer science courses
        \end{onecolentry}

        \vspace{0.2 cm}

        \begin{onecolentry}
            \textbf{Third Prize, Senior Design Project:} Awarded 3rd prize for a synchronized calendar project out of 100 entries
        \end{onecolentry}


    
    \section{Technologies}



        
        \begin{onecolentry}
            \textbf{Languages:} C++, C, Java, Objective-C, C\#, SQL, JavaScript
        \end{onecolentry}

        \vspace{0.2 cm}

        \begin{onecolentry}
            \textbf{Software:} .NET, Microsoft SQL Server, XCode, Interface Builder
        \end{onecolentry}


    

\end{document}