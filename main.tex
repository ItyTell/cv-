\documentclass[10pt, letterpaper]{article}


% Packages:
\input{packages}
% Settings:
\input{settings}

\begin{document}
    \newsavebox\ANDbox
    \sbox\ANDbox{$|$}

    \begin{header}
        \fontsize{25 pt}{25 pt}\selectfont Микола Коломієць

        \vspace{5 pt}

        \normalsize
        \mbox{Київ, Україна}%
        \kern 5.0 pt%
        \AND%
        \kern 5.0 pt%
        \mbox{\hrefWithoutArrow{mailto:nickkolomietsdooor@gmail.com}{nickkolomietsdooor@gmail.com}}%
        \kern 5.0 pt%
        \AND%
        \kern 5.0 pt%
        \mbox{\hrefWithoutArrow{tel:+380 672 570 278}{+380 672 570 278}}%
        \kern 5.0 pt%
        \AND%
        \kern 5.0 pt%
        \mbox{\hrefWithoutArrow{https://itytell.github.io/site/}{Portfolio site}}%
        \kern 5.0 pt%
        \AND%
        \kern 5.0 pt%
        \mbox{\hrefWithoutArrow{www.linkedin.com/in/mycola-kolomiiets-5350961b0}{linkedin}}%
        \kern 5.0 pt%
        \AND%
        \kern 5.0 pt%
        \mbox{\hrefWithoutArrow{https://github.com/ItyTell}{github}}%
    \end{header}

    \vspace{5 pt - 0.3 cm}

    \section{Освіта}
        
        \begin{twocolentry}{
            Вересень 2020 – Червень 2024
        }
            \textbf{Київський нацьональний університет ім. Тараса Шевченка}, факультет комп'ютерних наук, спеціальність прикладна математика \end{twocolentry}

        \vspace{0.10 cm}
        \begin{onecolentry}
            \begin{highlights}
                \item GPA: 89/100
            \end{highlights}
        \end{onecolentry}



    
    \section{Досвід}

        
        \begin{twocolentry}{
            Серпень – Жовтень 2023
        }
            \textbf{Степендіат програми "IRIS-HEP Fellow 2023"}, Прінстонський університет, CERN \end{twocolentry}

        \vspace{0.10 cm}
        \begin{onecolentry}
            Під час програми співпрацював з спеціалістами у галузях розробки програмного запезпечення та data science над проектом, пов'язаним з фізикою високих часток.
        \end{onecolentry}

    \section{Мова}
        

        \begin{samepage}
            \textbf{Англійська, рівень B2}


            Рівень B2, через невелику кількість помилок на ЗНО і на ЄВІ. Маю досвід роботи з іноземними колегами та викладачами. Більшість контенту що споживав 4 роки було англійською мовою, тому розуміння мови на високому рівні проте не було багато розмовної практики.


            \textbf{Українська мова, рідна}

        \end{samepage}

    \section{Навички}

            
        \begin{onecolentry}
            \begin{highlightsforbulletentries}


            \item Python


            В університеті я пройшов річний курс з Python. 
            Потім використовував його для автоматизації розрахунків з інших предметів і створювати особисті проекти, такі як телеграм-бот для надання певної статистики та просту 2-D гру на pygame. Останніми алгоритмами, які я реалізував у Python, були деякі алгоритми кластеризації: k-середні, нечіткий k-means (варіація попереднього) і DBScan. 
            Також використовував python для дипломної роботи з прогнозування шкідливих викидів у атмосферу. 

            \item Математика
            

            Я завершив навчання за спеціальністю «прикладна математика». 
            Я вивчав різні розділи математики, включаючи статистику, теорію ймовірності, чисельні методи та математичний аналіз. 


            \item Глибоке навчання 
            

            Я пройшов курс на Coursera «Спеціалізація глибокого навчання» від Deep Learning.AI і отримав сертифікат у 2023 році. 
            Тоді я погрався з набутими знаннями в Jupyter Notebook, створивши та навчання декількох нейронних мереж. 
            Застосовував нейронні мережі у дипломній роботі та в особистих проектах, також працював з нейронними мережами на степендіальній програмі.

            \item Git, Docker, Linux
            

            Використовую git для більшості своїх проектів та маю досвіт роботи з git та dicker після степендіальної програми. 
            Раніше мав ком'ютер на Linux, зараз використовую Linux переважно через WSL для застосування графічного процесора у tensorflow (основний комп'ютер на Windows).
            
            
            \item C++
            

            Я пройшов річний курс C++ в університеті. 
            Курс був більше про об'єктно-орієнтовану парадигму, але всі приклади та завдання були на С++. 


            \item SQL
            

            В університеті я пройшов піврічний курс SQL. 
            Під час курсу я навчився писати запити різної складності, в тому числі дуже складноі. 
            Курс базувався на Microsoft Access, хоча я пізніше застосував отримані знання, під час роботи над телеграм-ботом на Python, для якого я спочатку хотів використати SQLite, але зупинився на MySQL.
            

            \end{highlightsforbulletentries}
        \end{onecolentry}

\end{document}